%================= gummi hack for multiple bibliographies ==============
% *sigh*
\bgroup
\def\bibfile{decoherence.bib}
\def\otherbibfile{decoherence_paper_draft_1Notes.bib}
\def\gtmpdir#1{C:/Users/Jason Gross/gtmp/#1}

\def\execcommand#1{\message{#1^^J}\immediate\write18{#1}}

\edef\bibfilecopy{\gtmpdir{\bibfile}}
\edef\otherbibfilecopy{\gtmpdir{\otherbibfile}}

\execcommand{cp -v "\bibfile" "\bibfilecopy"}
\execcommand{cp -v "\otherbibfile" "\otherbibfilecopy"}

\execcommand{bibtex decoherence_paper_draft_1}
\execcommand{bibtex "\gtmpdir{\jobname}"}
\egroup
%======================================================================


%\documentclass[aps,prd,preprint]{revtex4}
\documentclass[aps,prd,final,twocolumn,10pt,longbibliography,nobibnotes]{revtex4-1}

% Note:  comment out one of the two documentclass commands, depending
% on whether you need the final or preprint version. It is most convenient
% to work in preprint mode, and switch to final only at the end.

\usepackage{graphicx}                            % Graphics package
\usepackage{amsmath} 
\usepackage{amssymb}

%\input{revtex-biblatex}

%\usepackage{xifthen}

\usepackage{mmap} % make PDF files generated by pdfLaTeX both searchable and copy-able in acrobat reader and other compliant PDF viewers
\setlength{\paperwidth}{8.5in}
\setlength{\paperheight}{11in}
\usepackage[plainpages=false,pdfpagelabels,unicode]{hyperref}
%\usepackage{enumerate}

%\usepackage{wrapfig}
%\usepackage{cancel}
%\usepackage{undertilde}

%\usepackage{stackrel}
%\usepackage{xcolor} % required to not duplicate email address?!........ but only when we're one page long.... or not

%\usepackage{mathtools}[2011/04/06]

%\usepackage{pgf,tikz}
%\usetikzlibrary{arrows}

%\usepackage{xfrac}[2010/02/02]


\DeclareMathOperator{\sech}{sech}
\newcommand{\bra}[1]{\left\langle #1 \right|}
\newcommand{\ket}[1]{\left|#1\right\rangle}
\newcommand{\braket}[2]{\left\langle#1 \middle| #2\right\rangle}
\newcommand{\rd}[1]{\,\mathop{\mathrm{d}#1}}

\newcommand{\defeq}{\equiv}%\mathrel{\mathrel{\mathop:}\mkern -1.2mu=}}%\coloneqq}%\stackrel{\mathrm{df}}{=}}%{\ensuremath{:=}}%


%%%%%%%%%%%%%%%


\newcommand{\mytitle}{Decoherence: An Explanation of Quantum Measurement}
\newcommand{\myauthor}{Jason~Gross}

\hypersetup{
  pdftitle={\mytitle},
  pdfauthor={\myauthor}
}

\begin{document}
\title{\mytitle}
\author{\myauthor}
\email{jgross@mit.edu}
\affiliation{Massachusetts Institute of Technology,
 77 Massachusetts Ave.,
Cambridge, MA 02139-4307}
\date{\today} 


\begin{abstract}
  \noindent
  The description of the world given by quantum mechanics is at odds with our classical experience.  Most of this conflict resides in the concept of ``measurement''.  After explaining the origins of the controversy, I will introduce decoherence as a way of describing classical measurements from an entirely quantum perspective.  I will discuss basis ambiguity and the problem of information flow in a system + observer model, and explain how introducing the environment removes this ambiguity via environmentally-induced superselection.  I intend to describe two toy models for decoherence to demonstrate measurement in a quantum system.  I intend to conclude with an explanation of the ``derivations'' of Born's Rule due to Everett, Gleason, and, most recently, Zurek, the last of which I believe uses decoherence.
\end{abstract}

\maketitle
\pagestyle{myheadings}
\markboth{\myauthor}{\mytitle}
\thispagestyle{empty}


\section{Introduction}
  Useful papers: \cite{zurek2003decoherence,Zurek1998}
  
  \subsection{The Problem: Classicality}
    ``Hilbert space is big''  Most of it is not classical.  This conflicts with what we seem to experience.
  
  \subsection{Interpretations of Quantum Mechanics}
    \subsubsection{Copenhagen}
    \subsubsection{Many Worlds}
    \allowbreak
  
  \subsection{The Role of The Environment}
    Talk about pointer states.
    
\allowbreak
\section{Environment, System, Observers}
  Talk about how to split the universe into enviornment, system, and observer.

\section{Measurement}

\section{Decoherence}
\allowbreak
  \subsection{Toy Model: \texorpdfstring{\texttt{c-not}}{c-not}}
    See \cite{zurek2003decoherence}.
   
  \allowbreak
  \subsection{Basis Ambiguity \& Direction of Information Flow}
    In \texttt{c-not}, it's not clear what the system is and what the observer is.
    
  \subsection{Environmentally Induced Superselection}
  
  
  \subsection{Toy Model: \texorpdfstring{\texttt{c-not}}{c-not} + Environment}
  
  \allowbreak
  \subsection{Toy Model for Decoherence}
    I'll probably use a system with spins, and spin-spin coupling with the environment, a la \cite{Cucchietti2005}.  I'll probably calculate decoherence time.

\allowbreak
\section{Deriving Born's Rule}
  Everett and Gleason independently(?) proved in \cite{Gleason1957, Everett1957} that the norm squared measure is the only possible probability measure on a Hilbert space of dimension $\ge 3$.  Zurek derives Born's Rule in \cite{zurek2003decoherence,Zurek2007,Zurek2005} from the stand-point of decoherence. 

\allowbreak
\section{What's Left?}
  Talk about remaining interpretational problems.
  
\section*{Acknowledgments}
  The author will be gratful to acknowledge Fakhri Zahedy (\href{mailto:fsz@mit.edu}{fsz@mit.edu}), his peer reviewer.
  
 
%The author is grateful to Jeffrey Goldstone for conversations on 
%reflectionless potentials.  This work is supported in part by funds 
%provided by the U.S. Department of Energy (D.O.E.) under cooperative 
%research agreement \#DF-FC02-94ER40818.


\nocite{*}
\bibliography{decoherence}

\end{document}
